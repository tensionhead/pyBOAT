In this section we aim to lay down the basic principles of Wavelet analysis as employed in our signal analysis tool. We mainly follow the authors of \cite{Torrence1998}, albeit are still omitting more mathematical subtleties to allow for a wider audience. Readers seeking a more formal introduction may find it here: \cite{Daubechies1992, Mallat1999} 

The historically oldest way of doing frequency analysis of periodic signals is the well known and ubiquitously used Fourier analysis. It's working principle is the decomposition of a signal into its harmonic components. A harmonic component is a Sine or Cosine with constant frequency. Mathematically the Fourier transform can be expressed as:

\begin{align}
  \label{Ftrafo}
  \mathcal{F}[s](f) &= \int_{-\infty}^{\infty} s(t)\;e^{-2\pi i f t} dt \\
  &= \int_{-\infty}^{\infty} s(t)\; \left[cos(\omega t) + i\;sin(\omega t) \right] dt
\end{align}

Here we used the Euler identity to express the complex Exponential as the sum of Cosine and Sine and $\omega = 2\pi f$. The result is the Fourier transformed signal $\mathcal{F}[s] = \hat{s}(f)$ which is a function of the frequency $f$ alone. 
This complex valued function $\hat{s}(f)$ has no direct physical meaning. It's absolute square however gives a real valued function, often denoted by the \textit{power spectral density} or just (Fourier-)spectrum of the signal $s$:
\begin{equation}
  P_\mathcal{F}(f) = |\hat{s}(f)|^2
\end{equation}
It describes the contribution of each harmonic component with fequency $f$ to the signal. 

The Fourier transform translates the signal from the \textit{time domain} into the \textit{frequency domain}: $\mathcal{F} : s(t) \rightarrow \hat{s}(f)$. As a corollary, all time-dependent information of the signal is lost in the frequency domain (see Figure \ref{fig1}a). Therefore Fourier analysis is best suited for \textit{stationary} signals, meaning here no varying frequencies over time. This is a situation often at least approximately found in fields like engineering (\cite{Smith1997}) or spectroscopy, but is rather rare in Biology. Fourier analysis nowadays comes in many forms an flavours, the following methods are all based on its concepts: Periodogramm, Welsh's method and Lomb-Scargle method.

Mathematically, decompositions of the form of equation \ref{Ftrafo} work by choosing a specific set of \textit{basis functions}. For the Fourier transform, these basis functions are the Sines and Cosines which have no \textit{localization in time} but are sharply \textit{localized in frequency}. Each harmonic component carries exactly one frequency effective everywhere in time. The idea to reach an optimal compromise between time and frequency localization goes back to Gabor (\cite{Gabor1946}). He introduced Gaussian modulated harmonic components:
\begin{align}
  \Psi(t) &= \pi^{-1/4} \; e^{-t^2/2} \; e^{i\omega_0 t}\\
  &= \pi^{-1/4} \; e^{-t^2/2} \; \left[cos(\omega_0 t) + i\; sin(\omega_0 t) \right]
\end{align}
This function is also known as \textit{Morlet Wavelet}. The basis function for time-frequency analysis are then generated by \textit{scaling} and \textit{translation}:
\begin{equation}
  \Psi_{s,\tau}(t) = s^{-1/2}\;\Psi \lt(\frac{t - \tau}{s}\rt)
\end{equation}
Varying the time localization $\tau$ slides the Wavelet left and right on the time axis. The scale $s$ changes the \textit{center frequency} of the wavelet according to $\omega_c = \omega_0/s$. Higher scales therefore generate Wavelets with lower center frequency. The Gaussian envelope suppresses the harmonic component with frequency $\omega_c$ farer away from $\tau$, therewith localizes the Wavelet in time. This frequency $\omega_c$ is conventionally taken as the Fourier equivalent (or pseudo-) frequency of a Morlet Wavelet with scale $s$. It's noteworthy to state that Wavelets in general are not as sharply localized in frequency as their harmonic counterparts, it's a tradeoff necessary for the gain in time localization. 

